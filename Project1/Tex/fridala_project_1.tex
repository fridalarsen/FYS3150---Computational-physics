\documentclass[notitlepage, reprint, nofootinbib]{revtex4-1}
\usepackage[utf8]{inputenc}

% Mathematics and symbols:
\usepackage{amsmath, gensymb, amsthm, physics, mhchem}
% Figures:
\usepackage{tikz, graphicx}

\usepackage[caption=false]{subfig}
% Other:
\usepackage{hyperref}


% Document formatting 
\setlength{\parskip}{1mm}
\setlength{\parindent}{0mm}

% Programming
\definecolor{codebackground}{rgb}{0.9,0.9,0.9}
\usepackage{listings}
\lstset{
	language=python,
	backgroundcolor=\color{codebackground},
	basicstyle=\scriptsize,
	aboveskip={1\baselineskip},
	columns=fixed,
	numbers=left,
	showstringspaces=false, 
	breaklines=true, 
	frame=single,
	showtabs=false,
	showspaces=false,
	keywordstyle=\color[rgb]{0,0,1},
	commentstyle=\color[rgb]{0.133,0.545,0.133}
	}

%\renewcommand{\thesubsubsection}{\alph{subsubsection})}

\hypersetup{
    colorlinks=true,
    linkcolor=blue,
    filecolor=magenta,      
    urlcolor=cyan,
}

\begin{document}
\title{FYS3150 - Project 1}
\author{Frida Larsen}
\maketitle

% Introduction
\section{Introduction}
{\color{red}{Reference code in github repository.}} Problem to solve
\begin{equation}\label{Poisson}-\dv{x} u(x)=f(x),\end{equation}
with 
\begin{equation}\label{info}x\in (0,1)\quad\text{and}\quad u(0)=u(1)=0.\end{equation}

% Theory
\section{Theory}
Approximation of second derivative of a general discretized function $g_i$: 
\begin{equation}\label{second_derivative} g_i'' = \frac{g_{i+1}+g_{i-1}-2g_i}{h^2}+\order{h^2}.\end{equation}
{\color{red}approx by removing order.} \\[2mm]
Discretized approximation $u\rightarrow u_i$ and $x\rightarrow x_i=ih$, where the step length $h$ is defined as
\begin{equation}\label{step-length}h=\frac{1}{n+1},\end{equation} 
where $n$ are the total number of points on the interval $\qty[x_0=0,\ x_{n+1}=1]$. The boundary conditions are then discretized as $u_0=u_{n+1}=0$.\\[2mm]
{\color{red}{fiks tekst i utledningen below}}\\[2mm]
Thus our equation, equation \ref{Poisson}, can be rewritten as
$$-\frac{u_{i+1}+u_{i-1}-2u_i}{h^2}=f_i.$$
By introducing $\tilde{b}_i=h^2 f_i$ and {\color{red}{inserting for $i$}} we get
$$\mqty(-u_0+2u_1-u_2 \\ -u_1+2u_2-u3 \\ \dots \\ -u_{n-1}+2u_n-2u_{n+1})=\mqty(b_1\\b_2\\b_3\\b_4).$$
Inserting we get
\begin{equation}\label{big_boy}\mqty( 2 & -1 & 0 & \dots & 0 & 0 \\ -1 & 2 & -1 & 0 &\dots&\dots \\ 0 & -1 & 2 & -1 & 0 & \dots \\ \dots & \dots & \dots & \dots &\dots &\dots \\ 0 & \dots & \dots & 0 & -1 & 2)\mqty(u_1\\u_2\\u_3\\\dots\\u_n)=\mqty(\tilde{b}_1\\ \tilde{b}_2\\ \tilde{b}_3\\ \tilde{b}_4 \\ \tilde{b}_5). \end{equation}
So we have 
\begin{equation}\label{small_boy}\vb{A}\vb{u}=\vb{\tilde{b}},\end{equation}
with the matrix $\vb{A}$ given by equation \ref{big_boy}.

% Method
\section{Method}



% Results
\section{Results}


% Discussion
\section{Discussion}

% Conclusion
\section{Conclusion}
















\end{document}